\documentclass[a4paper, 11pt]{report}

\usepackage{
	amssymb, 
	amsmath
}

\usepackage[paperwidth=8.5in,paperheight=11.0in,
  left=1.0in,right=1.0in,top=1.0in,bottom=1.0in,
  includefoot,heightrounded]{geometry}

\newcommand{\R}{\mathbb{R}}

\newcommand{\newchapter}[2]{
	\chapter{#1}
	\addtocontents{toc}{\vspace{.1in} \hspace{.25in} $\cdot$ #2 \par}
}

\DeclareMathOperator*{\argmin}{\arg\!\min}

\newcommand*{\titleTH}{\begingroup % Create the command for including the title page in the document
	\center
	\vspace*{\baselineskip} % Whitespace at the top of the page
	\vspace{2.5in}
	{\Huge\bfseries STA290}\\[\baselineskip] % First part of the title, if it is unimportant consider making the font size smaller to accentuate the main title
	{\Huge\texttt{Winter 2014}}\\[\baselineskip] % Main title which draws the focus of the reader
	{\Large \textit{Selected presentation materials}}\par % Tagline or further description
	\vspace*{3\baselineskip} % Whitespace at the bottom of the page
\endgroup}

\setlength{\parindent}{0pt}

%%%%%%%%%%%%%%%%%%%%%%%%%%%%%%%%%%%%%%%%%%%%%%%%%%%%%%%%%%%%%%%%
\begin{document}
%%%%%%%%%%%%%%%%%%%%%%%%%%%%%%%%%%%%%%%%%%%%%%%%%%%%%%%%%%%%%%%%
\titleTH %Title page command
\thispagestyle{empty}
%%%%%%%%%%%%%%%%%%%%%%%%%%%%%%%%%%%%%%%%%%%%%%%%%%%%%%%%%%%%%%%%
\tableofcontents
%%%%%%%%%%%%%%%%%%%%%%%%%%%%%%%%%%%%%%%%%%%%%%%%%%%%%%%%%%%%%%%%

 %BEGIN CONSTRUCTION OF CHAPTERS



%%%%%%%%%%%%%%%%%%%%%%%%%%%%%%%%%%%%%%%%%%%%%%%%%%%%%%%%%%%%%%%%
\newchapter{Orthogonal Projections for Fixed Vectors}{2/13/14}


For any two fixed vectors $z_1,z_2$ that are of equal dimension,
\begin{align*}
	\langle z_1,z_2\rangle = z_1\cdot z_2 = z_1^{T}z_2
\end{align*}

Suppose $M$ is an $m$-dimensional linear space that is a subset of $\R^d$. Consequently, define $M^{\perp}$ as the orthogonal compliment of $M$ that has dimension $d-m$ which can be defined with $\phi_k$ as mutually orthogonal vectors for $k = 1,...,d$ such that
\begin{align*}
	M &= span\{\phi_1,...,\phi_m\} \label{span} \\
	M^{\perp} & = span\{\phi_{m+1},...,\phi_d\} 
\end{align*}

Any vector $z \in \R^d$ can be uniquely represented as $x_1 + x_2 = z $ where $x_1 \in M$ and $x_2 \in M^{\perp}$, which is to say


\begin{align*}
	M \oplus M^{\perp} = \R^d
\end{align*}

\vspace{.1in}


For any $y \in \R^d$ notice that 
\begin{align*}
	y &= \sum\limits_{k=1}^d \langle y, \phi_k\rangle\phi_k \\
	||y||^2 &= \langle y,y\rangle = \sum\limits_{k=1}^d \langle y, \phi_k\rangle^2 
\end{align*}

\vspace{.1in}

and the projection of $y$ onto $M$ is defined as 
$P_My = \argmin_{w \in M} ||w-y||^2 $, which can be expressed as

\begin{align}
\boxed{P_My = \sum\limits_{k=1}^m \langle y, \phi_k\rangle\phi_k }
\end{align}

The following relation holds for the space $M$ and its orthogonal compliment $M^{\perp}$ \vspace{.1in}
\begin{align*}
\boxed{P_My \perp P_{M^{\perp}}y \qquad i.e. \qquad P_My \perp (y - P_My) \bigg.}
\end{align*}

%%%%%%%%%%%%%
%END CHAPTER%
%%%%%%%%%%%%%





%%%%%%%%%%%%%%%%%%%%%%%%%%%%%%%%%%%%%%%%%%%%%%%%%%%%%%%%%%%%%%%%
\newchapter{Projections for Gaussian Random Vectors}{2/13/14}

Assume a Gaussian setting where we consider $Y$ a dimension $d \times 1$ vector of values and $U$ as an orthogonal "rotation" matrix of dimension $d\times d$, where $I = U^{T}U$. 
\begin{align*}
	Y \sim N\left(0, \sigma^2 I_d\right)\quad \Rightarrow \quad UY \sim N\left(0, \sigma^2I_d\right)
\end{align*}


Several consequential results are expressed from the properties of fixed vectors with the orthogonal $\; \phi_1,...,\phi_d \;$ representation where  $\; M = span\{\phi_1,...,\phi_m\} \;$.

\begin{align}
	\boxed{\left(\begin{array}{c} \langle Y,\phi_1\rangle \\ \vdots \\ \langle Y,\phi_d\rangle \end{array}\right) \sim N(0,\sigma^2 I_d)}
\end{align}


\vspace{.1in}

\begin{align}
	\boxed{||P_MY||^2 = \sum\limits_{k=1}^m \langle Y,\phi_k\rangle^2 \quad \sim \quad \sigma^2 \chi^2_m}
\end{align}


\vspace{.1in}

\begin{align}
	\boxed{P_MY \text{ is independent of } (Y-P_MY) \bigg.} 
\end{align}

%%%%%%%%%%%%%
%END CHAPTER%
%%%%%%%%%%%%%







%%%%%%%%%%%%%%%%%%%%%%%%%%%%%%%%%%%%%%%%%%%%%%%%%%%%%%%%%%%%%%%%
\end{document}
%%%%%%%%%%%%%%%%%%%%%%%%%%%%%%%%%%%%%%%%%%%%%%%%%%%%%%%%%%%%%%%%

