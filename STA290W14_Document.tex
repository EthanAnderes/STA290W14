\documentclass[11pt]{report}

\usepackage{amsmath,amsthm,amssymb,epsf,eucal}
\usepackage{bm} %% Bold Math
\usepackage{graphicx,psfrag}
\usepackage{epstopdf}
\usepackage{verbatim}
\usepackage{enumerate}
\usepackage{multirow}
\usepackage{color}
\usepackage{setspace}
\usepackage{subfigure}
\usepackage{hyperref}

% \documentclass[a4paper, 11pt]{report}

% \usepackage{
% 	amssymb, 
% 	amsmath
% }

% \usepackage[paperwidth=8.5in,paperheight=11.0in,
%   left=1.0in,right=1.0in,top=1.0in,bottom=1.0in,
%   includefoot,heightrounded]{geometry}



\newtheorem{claim}{Claim}
\newtheorem{corollary}{Corollary}
\newtheorem{proposition}{Proposition}
\newtheorem{theorem}{Theorem}
\newtheorem{lemma}{Lemma}

\newcommand{\R}{\mathbb{R}}
\newcommand{\bs}{\boldsymbol}
\newcommand{\newchapter}[2]{
	\chapter{#1}
	\addtocontents{toc}{\vspace{.1in} \hspace{.25in} $\cdot$ #2 \par}
}

\DeclareMathOperator*{\argmin}{\arg\!\min}

\newcommand*{\titleTH}{\begingroup % Create the command for including the title page in the document
	\center
	\vspace*{\baselineskip} % Whitespace at the top of the page
	\vspace{2.5in}
	{\Huge\bfseries STA290}\\[\baselineskip] % First part of the title, if it is unimportant consider making the font size smaller to accentuate the main title
	{\Huge\texttt{Winter 2014}}\\[\baselineskip] % Main title which draws the focus of the reader
	{\Large \textit{Selected presentation materials}}\par % Tagline or further description
	\vspace*{3\baselineskip} % Whitespace at the bottom of the page
\endgroup}

%\setlength{\parindent}{20pt}

%%%%%%%%%%%%%%%%%%%%%%%%%%%%%%%%%%%%%%%%%%%%%%%%%%%%%%%%%%%%%%%%
\begin{document}
%%%%%%%%%%%%%%%%%%%%%%%%%%%%%%%%%%%%%%%%%%%%%%%%%%%%%%%%%%%%%%%%
\titleTH %Title page command
\thispagestyle{empty}
%%%%%%%%%%%%%%%%%%%%%%%%%%%%%%%%%%%%%%%%%%%%%%%%%%%%%%%%%%%%%%%%
\tableofcontents
%%%%%%%%%%%%%%%%%%%%%%%%%%%%%%%%%%%%%%%%%%%%%%%%%%%%%%%%%%%%%%%%

 %BEGIN CONSTRUCTION OF CHAPTERS



%%%%%%%%%%%%%%%%%%%%%%%%%%%%%%%%%%%%%%%%%%%%%%%%%%%%%%%%%%%%%%%%
%\newchapter{Coordinate free orthogonal projections for fixed vectors}{2/13/14}
\chapter{Abstract vector spaces}

\section{Coordinate free projections}


In this chapter we consider some abstract vector space, call it $V$, which has a norm $\|\cdot \|$ induced by an inner product $\langle \cdot, \cdot \rangle$ (i.e. $\|v\|^2 := \langle v, v\rangle\rangle$). The elements of $V$ are called vectors but since we are considering abstract spaces we do not necessarily have access to the ``coordinates'' of each vector. What we do have is the ability to construct an orthonormal basis of $V$, usually denoted something like $\phi_1,\phi_2, \ldots, \phi_d$, where $d$ is the dimension of $V$. Then each vector in $v$ has a decomposition in terms of these basis vectors. This note explores to computing things like norms and inner products and taking projections with these basis vectors. 



The idea is that $V$ will denote something like points in space. In physics, points in space should exist independently of what coordinate system one uses. In particular, the notion of distance and inner product exist {\emph before} we attach a coordinate system to this space. The following chapter talks about how to work with such vectors. The main story is that one can choose a particular orthonormal basis and then the coefficients of the basis decomposition behave exactly as regular coordinates in $\mathbb{ R}^d$. Moreover, if one wants to project a vector in $V$ to a linear subspace $M\subset V$ then by choosing the basis vectors appropriately one can easily perform the desired projection by simply truncating the basis representation.


\begin{claim}[{\bf Coefficients are coordinates}] Let $\phi_1,\ldots, \phi_d$ be an orthonormal basis for $V$. Then any $v\in V$ has a unique representation 
\[\boxed{ v = \sum_{k=1}^d c_k \phi_k,\,\,\text{where $c_k = \langle v, \phi_k\rangle$}.} \]
 Moreover, these basis coefficients behave exactly like coordinates in regular Euclidean space so that if $v = \sum_{k=1}^n c_k \phi_k$ and $w = \sum_{k=1}^n d_k \phi_k$, then
\[
\boxed{\langle v,w\rangle = \sum_{k=1}^d c_kd_k.}
\]
In particular,  $\| v \|^2 = \sum_{k=1}^d c_k^2$.
\end{claim}



Now suppose  $M\subset V$ is an $m$-dimensional linear subspace (the zero vector should be in here). We can define $M^{\perp}$ to be the set of all vectors in $V$ which are orthogonal to  every vector in $M$. In this case we can write $M \oplus M^{\perp} = V$ to signify that every $v\in V$ has a unique decomposition $v_M + v_M^\perp$ where $v_M\in M$ and $v_M^\perp \in M^\perp$. Often, in statistics, one needs to project a vector $v\in V$ to a subspace $M$. This operation is denoted $P_M v$ and is technically defined as $P_Mv := \argmin_{w \in M} ||w-v||^2 $. The following theorem shows that with a judicious choice of your orthonormal basis this projection is easily calculated

\begin{claim}[{\bf Projections are easy}]\label{proj1} If one constructs the orthonormal basis $\phi_1,\ldots, \phi_d$ of $V$ in such a way that
\begin{align*}
	M &= span\{\phi_1,...,\phi_m\} \label{span} 
\end{align*}
then for any vector $v=\sum_{k=1}^d c_k \phi_k$ the projection to $M$ is computed by truncating the decomposition to m:
\begin{align}
\boxed{P_Mv = \sum\limits_{k=1}^m c_k\phi_k }.
\end{align}
Moreover, since $P_Mv$ must be orthogonal to  $P_{M^{\perp}}v$, and $P_{M^{\perp}}v = v - P_Mv$, we have that 
\[\boxed{P_Mv \perp (v - P_Mv).}\]

\end{claim}



The following is a simple consequence of the above claim, but it will be important later so we state it here
\begin{corollary}\label{cor1}
If $M$ is a linear subspace of $V$ and $v,w \in V$ then
\[\langle v, P_M w\rangle = \langle P_M v, P_M w\rangle = \langle P_M v,  w\rangle. \]  
\end{corollary}




% For any two fixed vectors $z_1,z_2$ that are of equal dimension,
% \begin{align*}
% 	\langle z_1,z_2\rangle = z_1\cdot z_2 = z_1^{T}z_2
% \end{align*}

% Suppose $M$ is an $m$-dimensional linear space that is a subset of $\R^d$. Consequently, define $M^{\perp}$ as the orthogonal compliment of $M$ that has dimension $d-m$ which can be defined with $\phi_k$ as mutually orthogonal vectors for $k = 1,...,d$ such that
% \begin{align*}
% 	M &= span\{\phi_1,...,\phi_m\} \label{span} \\
% 	M^{\perp} & = span\{\phi_{m+1},...,\phi_d\} 
% \end{align*}

% Any vector $z \in \R^d$ can be uniquely represented as $x_1 + x_2 = z $ where $x_1 \in M$ and $x_2 \in M^{\perp}$, which is to say


% \begin{align*}
% 	M \oplus M^{\perp} = \R^d
% \end{align*}

% \vspace{.1in}


% For any $y \in \R^d$ notice that 
% \begin{align*}
% 	y &= \sum\limits_{k=1}^d \langle y, \phi_k\rangle\phi_k \\
% 	||y||^2 &= \langle y,y\rangle = \sum\limits_{k=1}^d \langle y, \phi_k\rangle^2 
% \end{align*}

% \vspace{.1in}

% and the projection of $y$ onto $M$ is defined as 
% $P_My = \argmin_{w \in M} ||w-y||^2 $, which can be expressed as

% \begin{align}
% \boxed{P_My = \sum\limits_{k=1}^m \langle y, \phi_k\rangle\phi_k }
% \end{align}

% The following relation holds for the space $M$ and its orthogonal compliment $M^{\perp}$ \vspace{.1in}
% \begin{align*}
% \boxed{P_My \perp P_{M^{\perp}}y \qquad i.e. \qquad P_My \perp (y - P_My) \bigg.}
% \end{align*}

%%%%%%%%%%%%%
%END CHAPTER%
%%%%%%%%%%%%%





%%%%%%%%%%%%%%%%%%%%%%%%%%%%%%%%%%%%%%%%%%%%%%%%%%%%%%%%%%%%%%%%
\section{Projections for Gaussians}


If we are working in an abstract vector space $V$, what do we even mean by a Gaussian vector in $V$? A random vector $Y$ is said to be Gaussian if $\langle v,Y\rangle$ is a Gaussian random variable for all $v\in V$. Now we want some notion of $Y\sim \mathcal N(0,\sigma^2 I_d)$. To be able to say $Y\sim \mathcal N(0,\sigma^2 I_d)$ we simply require that there exists a orthonormal basis representation $Y= \sum_{k=1}^d d_k \psi_k$ where $d_1,\ldots, d_d$ are iid $\mathcal N(0,\sigma^2)$. 

A fundamental fact about independent mean zero Gaussian random variables is that they are invariant under rotations. In particular if $W \sim \mathcal N\left(0, \sigma^2 I_d\right)$ where $W$ is a random vector in Euclidean space then $UW \sim \mathcal N\left(0, \sigma^2I_d\right)$ for any orthogonal rotation matrix ($U$ is a rotation if  $I = U^{t}U$). In fact, independent mean zero Gaussians make up the only multivariate distribution with independent coordinates that is rotationally invariant. 
You can think of left multiplication by a rotation matrix as simply changing basis. Therefore if  $Y$ is an abstract random vector that satisfies $Y\sim \mathcal N(0,\sigma^2 I)$, then {\em any} basis decomposition of $Y$ should give iid $\mathcal N(0,\sigma^2)$ coefficients. 

\begin{claim}
Suppose $Y$ is an abstract Gaussian random vector in $V$ which satisfies $Y\sim \mathcal N(0,\sigma^2 I)$. If $\phi_1,\ldots, \phi_d$ is a orthonormal basis of $V$ then 
 
 \[ 
 \boxed{Y = \sum_{k=1}^d c_k \phi_k \text{ implies } c_1,\ldots, c_d \text{ are iid $\mathcal N(0,\sigma^2)$.}}
 \]
\end{claim}

Once we have the above theorem the following corollary is easy to prove.

\begin{corollary}\label{cor2}
Suppose $\phi_1,\ldots, \phi_d$ is a orthonormal basis of $V$ and $Y\sim \mathcal N(0,\sigma^2 I_d)$. Suppose $M$ is an $m$-dimensional linear subspace of $V$ which is spanned by $\phi_1,\ldots, \phi_m$. If $Y = \sum_{k=1}^d c_k \phi_k$ and $v = \sum_{k=1}^d v_k \phi_k$ then the following three statements are true:
\begin{enumerate}[(i)]
\item $||P_MY||^2 = c_1^2+\cdots c_m^2 \sim \sigma^2 \chi^2_m$;
\item $P_MY$ is independent of  $Y-P_MY$;
\item $var(\langle v,Y\rangle) = var( v_1 c_1+\cdots v_dc_d) =  \sigma^2 \| v \|^2$.
\end{enumerate}
\end{corollary}

Just as a simple example of power and simplicity of these statements notice that one can easily recover the fact that $\overline Y$ and $S^2$ are independent.


%%%%%%%%%%%%%%%%%%%%%%%%%%%%%%%%%%%%%%%%%%%%%%%%%%%%%%%%%%%%%%%%%%%%%%%%%%%%%%%%%%

\section{Application to linear models}

The basic coordinate free linear model is that the random vector $Y$ satistfies
\[
Y = \mu + Z
\]
where $Y, \mu$ and $Z$ all take values in a vector space $V$, $Z\sim \mathcal N(0,\sigma^2I)$ within the vector space $V$ and $\mu$ is assumed to be in some linear subspace $M\subset V$. The Gauss-Markov theorem shows that the optimal estimate of $\mu$ is the projection of $Y$ onto $M$, i.e. $\hat\mu = P_M Y$. 

\subsection{Gauss-Markov}
\begin{claim}[Gauss-Markov]
For any $v\in M$, the minimium variance linear unbiased estimate of $\langle v,\mu\rangle$ is given by $\langle v,\hat\mu\rangle$ where $\hat\mu := P_M Y$. 
\end{claim}


\begin{proof}
There are two main facts in the proof of the Gauss-Markov Theorem. First, notice 
\begin{align}
\langle v,\hat\mu\rangle &= \langle P_Mv,P_M Y \rangle = \langle P_Mv,Y \rangle \label{GMform}
\end{align}
which follows by Corollary \ref{cor1}. 
The second fact is to simply unravel what it means for a linear estimate, $\langle w,Y\rangle$ of $\langle v,\mu\rangle$ to be unbiased. Notice that $E\langle w,Y\rangle = \langle w, \mu\rangle= \langle P_M w, \mu\rangle$ so the unbiased constrant becomes
\begin{equation}
\langle P_Mw,\, \mu\rangle =  \langle P_M v,\, \mu\rangle\text{ for all $\mu\in M$}.
\end{equation} 
Now this implies that $P_Mw = P_Mv$ because the above formulat implies $\langle  P_Mw - P_Mv, \mu \rangle = 0$ then plug in $\mu = P_Mw - P_Mv\in M$. To summarize
\begin{quote}
\em
A necessary and sufficient condition that $\langle w, Y\rangle$ be a linear unbiased estimate of $\langle v, \mu\rangle$ is that $P_M w = P_M v$.
\end{quote}
By (\ref{GMform}) we know that the Gauss-Markov estimate is a linaer unbisaed estimate.
Now it is easy to show the Gauss-Markov estimator has minimum variance among linaer unbisaed estimates
\begin{align*}
\text{var}(\langle w, Y\rangle) &= \text{var}(\langle w, Z\rangle) \\
 &= \sigma^2\|w \|^2, \quad\text{by \ref{cor2}} \\
 &= \sigma^2\|P_Mw \|^2 + \sigma^2\|P_M^\perp w \|^2\\
 &\geq \sigma^2\|P_Mw \|^2 \\
 & = \sigma^2\|P_M v \|^2,\quad\text{if $\langle w, Y\rangle$ is unbiased} \\
 & = \text{var}(\langle P_M v, Y\rangle) \\
 & = \text{var}(\langle v, \hat\mu\rangle).
\end{align*}

\end{proof}


\subsection{Regression}
Notice that the Gauss-markov theorem has implications for ANOVA, regression and most other linear statistical models. In this seciton we explore the Gauss-Markov theorem in the context of regression. The standard regression setup is 
\[
Y = X \beta + Z
\]
where $X$ is the design maktrix with $p$ columns  and $\beta$ is a $p$ dimensional coefficient vector of unkonwns and $Z\sim \mathcal N(0,\sigma^2 I)$. This is indeed equivalent to our basic linear model
\[Y = \mu + Z \]
where now $\mu$ is assumed to be in the linear space $M:= \{ \beta_1 x_1+\cdots \beta_p x_p:\beta_k\in \Bbb R\}$ where $x_k$ is the $k^\text{th}$ column of $X$. 

To reconsile the OLS estimate  $\hat\beta = (X^tX)^{-1}X^t Y$ and the Gauss-markov estimate $\hat\mu = P_M Y$ simply notice that the OLS estimate of $\mu$ is simply $X\hat \beta = X(X^tX)^{-1}X^t Y$ which is the same as $P_M Y$. In paricular, $X(X^tX)^{-1}X^t$ is the coordinate-matrix form is $P_M$. This is easy to see when one adds the additional assumption that the columns of $X$ are orthonormal. In this case we know $X^tX = I$ and from Claim \ref{proj1} we know how to do the projection:
\begin{align*}
P_M Y &= \sum_{k=1}^p \langle Y, x_k\rangle x_k \\
&= XX^tY,\quad\text{since $X^tY$ gives $\langle Y, x_k\rangle$}\\
&= X(X^tX)^{-1}X^tY \\
&= X\hat \beta.  
\end{align*}

Now a few nice facts from regression automatically follow from the basic facts of projection. In partucular we know that $P_M$ is independent of $Y-P_MY$ from Corollary \ref{cor2}. This translates to the fact that 
\begin{quote}
\em The estimated residuals $Y - X\hat \beta$ and the esimated mean $X\hat \beta$ are independent (and orthogonal as vectors).
\end{quote}
We also know that when $\mu$ is the zero vector (i.e. there is no signal) then $\|\hat \mu \|^2 = \|X\hat\beta \|^2 = \|P_M Y \|^2\sim  \sigma^2 \chi^2_p$. In a similar manner it is also clear that $\|Y - \hat \mu \|^2 \sigma^2 \chi^2_{n-p}$. This gives the classic $F$ test for $H_0:\mu=0$
\begin{quote}
\em
Under the null hypothesis that $\mu = 0$ the random variable $F:= {\|\hat \mu \|^2}/{\|Y - \hat \mu \|^2}$ has the same distrubtion as ${Z_1}/{Z_2}$ where $Z_1\sim \chi^2_p$ and $Z_2\sim \chi^2_{n-p}$ are indepednent random variables.
\end{quote}
Note that the independence again comes from the fact that projections of spherical Gaussians are independent of their residuals. This way of viewing the $F$ test is very natural. ${\|\hat \mu \|^2}/{\|Y - \hat \mu \|^2}$ measures the size of $\| \hat mu\|^2$ normalized by and estimate of $\sigma^2$ so $F$ is unit free. One can also understand $\| \hat mu\|^2$ as quantifying the explained varability in $Y$ and $\|Y - \hat \mu \|^2$ as the unexplained variability in $Y$.

\subsection{Likelihood ratio statistic}


\section{Gaussian $E(X_0|X_1,\ldots, X_n)$ as projection}


Most of what we have done with our absract vector space results could have been don't just as easy (albite a bit more messy) with coordinates. This hides the true power and generality of the abstract projection results. In this section we analyze projection of random variables which have no natural coordinate vector representation. In paricular we show that Gaussian conditional expectation is simply linear projection onto the linear space of the observations. 


%%%%%%%%%%%%%%%%%%%%%%%%%%%%%%%%%%%%%%%%%%%%%%%%%%%%%%%%%%%%%%%%
\chapter{Fubini}

\chapter{Ideas Involving Bayesian Risk}

In this chapter we will explore some ideas involved in finding a Bayes rule (or Bayes estimator), and its associated Bayes risk under a specific loss function. 


Consider the family of Poisson distributions $\mathcal{P}(\lambda)$, $\lambda>0$, with p.m.f. given by 
$$p(x|\lambda) = e^{-\lambda}\frac{\lambda^x}{x!}, \quad x=0,1,2, \dots, $$
where we are interested in estimating $\lambda$ under the loss function
\begin{equation} \label{eq:loss}
l(\lambda,\delta(X)) = \frac{(\lambda-\delta(X))^2}{\lambda}.
\end{equation}
	
\begin{claim}
The Gamma family of distributions forms a conjugate family of priors for $\lambda$.
		
Specifically, let $\lambda \sim Gamma(\alpha,\beta)$ be a prior on $\lambda$ with prior hyper parameters $\theta_0 = (\alpha,\beta)$. Then the posterior is $\lambda|x \sim Gamma\left(\alpha+x, \frac{\beta}{\beta+1}\right)$, with posterior hyper parameters $\theta_1 = \left(\alpha+x, \frac{\beta}{\beta+1}\right)$. 
\end{claim}
		
Under this Gamma prior, a natural question is to find the associated Bayes estimator, or Bayes rule, $\widehat{\lambda}_\pi$.

By definition, a Bayes estimator $\delta^*(X)$ is a decision rule such that it minimizes the posterior risk; i.e.  
$$r(\delta^*(x)|x) = \inf_{\delta}\; r_\pi(\delta|x) $$
where $r_\pi(\delta|x)$ is defined as 
\begin{equation}\label{eq:postrisk}
r_\pi(\delta|x) = \mathbb{E}_{\lambda|x}(l(\lambda,\delta(X))|X=x) = \int_\Lambda l(\lambda,\delta(X)) \pi(\lambda|x) \; d\lambda,
\end{equation}
and the expectation is taken with respect to the posterior distribution of $\lambda$ given $X=x$, and $\lambda$ is distributed as $\pi$. 

In order to find this desired rule, $\widehat{\lambda}_\pi = \delta^*(X)$, and subsequently its associated Bayes risk, there are a number of ways to go about performing the actual computations. Our goal in this note is to show that not all of these methods are ``created equal'' in terms of computational time, which is especially relevant when taking an in-class exam. \newline

 
\underline{\textbf{Method I:}} Finding $\widehat{\lambda}_\pi$ directly. \newline
Using equation (\ref{eq:postrisk}), we can substitute in our loss function (\ref{eq:loss}), and arrive at  
\begin{align*}
r_\pi(\delta|x) 
& = \mathbb{E}_{\lambda|x}\left(\frac{(\lambda-\delta)^2}{\lambda}\right )  = \mathbb{E}_{\lambda|x}\left(\lambda\right) - 2\delta + \delta^2\mathbb{E}_{\lambda|x}\left(\frac{1}{\lambda}\right ) 
\end{align*}
We can find the infimum of this Bayes risk over all decision rules $\delta$ by taking the derivative with respect to $\delta$, and setting equal to 0.
		
$$\frac{\partial r_\pi(\delta|x)}{\partial \delta} =  - 2\delta + \delta^2\mathbb{E}_{\lambda|x}\left(\frac{1}{\lambda} \right )$$
$$\implies \quad \delta^*(X) = \frac{1}{\mathbb{E}_{\lambda|x}\left(\frac{1}{\lambda} \right )}$$
This expected value, taken under the posterior distribution, is equal to $\frac{\beta+1}{(\alpha +x-1)\beta}$
\begin{align*}
\mathbb{E}_{\lambda|x}\left(\frac{1}{\lambda} \right) 
& = \int \frac{1}{\lambda} \pi(\lambda|x) \; d\lambda 
= \int \frac{1}{\lambda} \frac{1}{\Gamma(\alpha+x)\left(\frac{\beta}{\beta+1}\right)^{\alpha+x}}\lambda^{\alpha+x-1}e^{-(\frac{\beta+1}{\beta})\lambda} \; d\lambda \\
& = \frac{\Gamma(\alpha+x-1)\left(\frac{\beta}{\beta+1}\right)^{\alpha+x-1}}{\Gamma(\alpha+x)\left(\frac{\beta}{\beta+1}\right)^{\alpha+x}}\int\frac{\lambda^{(\alpha+x-1)-1}e^{-(\frac{\beta+1}{\beta})\lambda}}{\Gamma(\alpha+x-1)\left(\frac{\beta}{\beta+1}\right)^{\alpha+x-1}} \; d\lambda \\
& = \frac{1}{(\alpha+x-1)\left(\frac{\beta}{\beta+1}\right)}
\end{align*}		
Thus, 
$$\delta^*(X) = \widehat{\lambda}_\pi =\frac{(\alpha +x-1)\beta}{\beta+1} $$

\underline{\textbf{Method II:}} A second method for finding this Bayes rule utilizes the following well-known theorem, and a subsequent corollary. 

\begin{theorem}
Given a squared-error loss function $l(\lambda, \delta(X)) = (\lambda - \delta(X))^2$, and a prior $\pi$ on $\lambda$, the Bayes rule is found by taking the expected value of the associated posterior distribution. i.e. $\widehat{\lambda}_\pi = E_{\lambda|x}(\lambda)$. 
\end{theorem}

\begin{corollary}
Let $\omega >0$ be a positive `weight'. Then, for the so-called `weighted squared-error` loss function $l(\lambda, \delta(X)) = \frac{(\lambda - \delta(X))^2}{\omega}$. The Bayes rule is found by solving the following integral:
$$\widehat{\lambda}_\pi = \int_\Lambda\frac{(\lambda - \delta(X))^2}{\omega} \pi(\lambda|x) d\lambda$$
where $\pi$ is a prior on $\lambda$, and $\pi(\lambda|X)$ is the density of the associated posterior distribution. 
\end{corollary}			

For our loss function given in (\ref{eq:loss}) we recognize that we have a ``weighted'' square-error loss. Thus, we identify $\omega = \lambda$, and write  
$$\widehat{\lambda}_\pi = \int_\Lambda\frac{(\lambda - \delta(X))^2}{\lambda} \pi(\lambda|x) d\lambda$$
Now, since our posterior distribution is a gamma with parameters $\left(\alpha+x, \frac{\beta}{\beta+1}\right)$, then taking this posterior density and dividing by $\lambda$ gives a density of a gamma distribution with parameters $\left(\alpha+x-1, \frac{\beta}{\beta+1}\right)$. It follows that the Bayes rule is simply the expected value of this newly-formed gamma distribution. i.e.
$$\widehat{\lambda}_\pi = \frac{(\alpha +x-1)\beta}{\beta+1}.$$
		
Next, we are interested in finding the Bayes risk of this Bayes rule. i.e. $r(\pi, \widehat{\lambda}_\pi)$. 
		
The Bayes risk for any decision rule $\delta$ is defined as 
		$$r(\pi, \delta) = \mathbb{E}_\lambda(R(\lambda,\delta)) = \mathbb{E}_\lambda\left[\mathbb{E}_{x|\lambda}\theta\left(l(\lambda, \delta)\right) \right]$$
		or, by explicitly writing out the expectations,
		$$r(\pi, \delta) = \int R(\theta,\delta) \pi(\lambda)\; d\lambda = \int\left[ \int l(\lambda,\delta)p(x|\lambda) \;dx\right] \pi(\lambda)\; d\lambda$$
		

Note that with $\widehat{\lambda}_\pi = (\alpha +x-1)\frac{\beta}{\beta+1}$, we can write	
$$\widehat{\lambda}_\pi = \gamma x + (\alpha-1)\gamma,\quad \text{where} \quad \gamma = \frac{\beta}{\beta+1}.$$


\underline{\textbf{Method I:}} Compute directly \newline
		\begin{align*}
		r\left(\pi, \widehat{\lambda}\right)
		&  = \mathbb{E}_\lambda\left[\mathbb{E}_{x|\lambda}\left(l(\lambda, \widehat{\lambda})\right) \right] 
		   = \mathbb{E}_\lambda\left[\mathbb{E}_{x|\lambda}\left(\frac{(\lambda-\widehat{\lambda})^2}{\lambda}\right) \right] \\
		&  = \mathbb{E}_\lambda\left[\mathbb{E}_{x|\lambda}\left(\lambda - 2\widehat{\lambda} + \frac{1}{\lambda}\widehat{\lambda}^2 \right) \right] 
		   = \mathbb{E}_\lambda\left[\lambda - 2\mathbb{E}_{x|\lambda}\left(\widehat{\lambda}\right) + \frac{1}{\lambda}\mathbb{E}_{x|\lambda}\left(\widehat{\lambda}^2\right)  \right] \\
		& = \cdots
		\end{align*}
This method quickly becomes a computational nightmare. \newline 


\underline{\textbf{Method II:}} Compute using the MSE: $\mathbb{E}(\lambda-\widehat{\lambda})^2 = Var(\widehat{\lambda}) + [E(\widehat{\lambda}) -\lambda]^2$ 

\begin{align*}
r\left(\pi, \widehat{\lambda}\right)
& = \mathbb{E}_\lambda\left[\mathbb{E}_{x|\lambda}\left(l(\lambda, \widehat{\lambda})\right) \right] 
= \mathbb{E}_\lambda\left[\frac{1}{\lambda}\left(Var_{x|\lambda}(\widehat{\lambda}) + [\mathbb{E}_{x|\lambda}(\widehat{\lambda}) -\lambda]^2 \right) \right] \\
& = \mathbb{E}_\lambda\left[\frac{1}{\lambda}\left(\gamma^2\lambda + [\gamma\lambda + (\alpha-1)\gamma-\lambda]^2 \right) \right] \\
& = \mathbb{E}_\lambda\left[\frac{1}{\lambda}\left(\gamma^2\lambda + \lambda^2(\gamma-1)^2 + 2\lambda\gamma(\gamma-1)(\alpha-1) + \gamma^2(\alpha-1)^2 \right) \right] \\
& = \mathbb{E}_\lambda\left[\gamma^2 + \lambda\frac{\gamma^2}{\beta^2} - 2\frac{\gamma^2}{\beta}(\alpha-1) + \frac{1}{\lambda}\gamma^2(\alpha-1)^2  \right] \\ 		
& = \gamma^2 + \alpha\beta\frac{\gamma^2}{\beta^2} - 2\frac{\gamma^2}{\beta}(\alpha-1) + \frac{1}{(\alpha-1)\beta}\gamma^2(\alpha-1)^2 \\   
& = \gamma^2 + \alpha\frac{\gamma^2}{\beta} - \frac{\gamma^2}{\beta}(\alpha-1)   		
= \gamma^2  + \frac{\gamma^2}{\beta}   		
= \gamma^2\left(1  + \frac{1}{\beta}\right)   	
= \frac{\beta}{\beta+1} \quad \qed 
\end{align*}
where we used the following:
$$Var_{x|\lambda}(\widehat{\lambda}) = \gamma^2\lambda, \quad  \mathbb{E}_{x|\lambda}(\widehat{\lambda}) = \gamma\lambda +(\alpha-1)\gamma, \quad \gamma-1 = \frac{-\gamma}{\beta} $$
\underline{\textbf{Method III:}} Compute by swapping expectations. \newline
Switch the order of expectations, which is allowed by applying Bayes Theorem to change from the prior to the posterior, and Fubini's Theorem which allows us to swap the order of integration.   
$$r(\pi, \delta) =  \int\left[ \int l(\lambda,\delta)p(x|\lambda) \;dx\right] \pi(\lambda)\; d\lambda = \int\left[ \int l(\lambda,\delta)\pi(\lambda|x) \;d\lambda\right] p(x)\; dx =  \mathbb{E}_x\left[\mathbb{E}_{\lambda|x}\left(l(\lambda, \widehat{\lambda})\right) \right]$$
Then, 
\begin{align*}
r\left(\pi, \widehat{\lambda}\right)
& = \mathbb{E}_x\left[\mathbb{E}_{\lambda|x}\left(l(\lambda, \widehat{\lambda})\right) \right] 
= \mathbb{E}_x\left[\mathbb{E}_{\lambda|x}\left(\lambda - 2\widehat{\lambda} + \frac{1}{\lambda}\widehat{\lambda}^2 \right) \right] \\
& = \mathbb{E}_x\left[\widehat{\lambda} + \frac{\beta}{\beta+1} - 2\widehat{\lambda} + \frac{1}{\widehat{\lambda}}\widehat{\lambda}^2  \right] 
= \mathbb{E}_x\left[\frac{\beta}{\beta+1}  \right] \\
& = \frac{\beta}{\beta+1}  \qed
\end{align*}
where we used 
$$\mathbb{E}_{\lambda|x}\left(\lambda\right) = \frac{(\alpha +x)\beta}{\beta+1} = \frac{(\alpha +x - 1)\beta}{\beta+1} + \frac{\beta}{\beta+1} = \widehat{\lambda} + \frac{\beta}{\beta+1},\text{ and } \mathbb{E}_{\lambda|x}\left(\frac{1}{\lambda}\right) = \frac{1}{\widehat{\lambda}}. $$



%%%%%%%%%%%%%%%%%%%%%%%%%%%%%%%%%%%%%%%%%%%%%%%%%%%%%%%%%%%%%%%%
\end{document}
%%%%%%%%%%%%%%%%%%%%%%%%%%%%%%%%%%%%%%%%%%%%%%%%%%%%%%%%%%%%%%%%

